\section{Additional functions}
\label{applications}
This section briefly describes the additional applications included in
the CaTS software package. Some of these applications contains a help
file. To view this you should use option {\tt -h}. Example:
\begin{verbatim}
./cats_interactive -h
\end{verbatim}

\subsection{The interactive mode}
\label{cats_interactive}
The program {\tt cats\_interactive} makes it possible to walk in the
Gr\"obner fan interactively.  You start the program with
\begin{verbatim}
./cats_interactive
\end{verbatim}
Now type in a matrix/point configuration. Example: {\tt
  ((1)(2)(3)(4)).} The program computes a reduced Gr\"obner basis with
respect to reverse lexicographic order with $a>b>c>d> \ldots $ where
$a,b,c,d$ are the variables of the toric ideal.

The program will print this reduced Gr\"obner basis, the radical of
its initial ideal, the standard pairs of its initial ideal and the
facets of the Gr\"obner cone.  You now have a few options:
\begin{itemize}
\item Type a number <enter> to flip the facet binomial indexed by that
  number.
\item Type 'c' <enter> to test for Cohen Macaulayness of the initial ideal.
\item  Type 'b' <enter> to take a step back in the path.
\end{itemize}
The program is useful when looking for initial ideals without embedded
primes --- we want to flip a facet binomial leading to an initial
ideal with fewer standard pairs as we expect this initial ideal to be
closer to an initial ideal without embedded primes (there is no
theorem to this effect, just an empirical observation). To easily find
such a facet, the facets have been marked with the number of standard
pairs for the ideal they flip to. Also the number of standard pairs
for the current initial ideal, the facet we came from and whether a
facet binomial flips to a different radical/triangulation is shown.

\subsection{Enumerating lattice points in polytopes}
\label{cats_fiber}
The program {\tt cats\_fiber} can enumerate the fiber of the map
$$\phi \, : \, \mathbb N^n \rightarrow \mathbb Z^d \,\, : \,\, u
\mapsto Au$$
for reasonably small examples. The input is a matrix of
the form described in section \ref{input} followed by a vector in the
fiber. Example:
\begin{verbatim}
./cats_fiber
\end{verbatim}
with input:
\begin{verbatim}
{(3)(4)(5)}
(0,0,3)
\end{verbatim}
gives output:
\begin{verbatim}
{
(0,0,3),
(2,1,1),
(1,3,0),
(5,0,0)
}
\end{verbatim}
Thus $\phi^{-1}(15) = \{(0,0,3), (2,1,1), (1,3,0), (5,0,0)\}$ when $A
= (3\,4\,5)$ --- in other words, there are four ways of writing $15$
as a sum of numbers from the set $\{3,4,5\}$:
$15=5+5+5=3+3+4+5=3+4+4+4=3+3+3+3+3$.

\subsection{VectorList2MonomialList}
\label{cats_vectorlist2monomiallist}
The program {\tt cats\_vectorlist2monomiallist} takes a list of
vectors and transforms it into a list of monomials. It is usually used
in combination with {\tt cats\_fiber}.
\noindent
Example:
\begin{verbatim}
./cats_fiber | ./cats_vectorlist2monomiallist
\end{verbatim}
with input:
\begin{verbatim}
{(3)(4)(5)}
(0,0,3)
\end{verbatim}
gives output:
\begin{verbatim}
(c^3,
a^2*b*c,
a*b^3,
a^5
)
\end{verbatim}

\subsection{IsDeltaNormal}
\label{cats_isdeltanormal}
The program {\tt cats\_isdeltanormal} can check if a point
configuration has a regular triangulation $\Delta$ for which it is
$\Delta$-normal with respect to the lattice generated by the points
\cite{gomoryips}. The program does the following
\begin{itemize}
\item computes a lattice basis of the generated lattice using the
  LLL-algorithm \cite{lenstra}
\item rewrites each of the input vectors as an integer combination of
  the lattice basis elements.
\item computes all regular triangulations of the new point
  configuration using TOPCOM \cite{topcom}
\item for each triangulation $\Delta$ it tests if the point
  configuration is $\Delta$-normal. This is done using {\tt normaliz}
  \cite{normaliz} on each triangle in $\Delta$.
\end{itemize}
Rewriting the points in another basis allows us to check normality of
each triangle with respect to $\Z^n$ instead of the lattice generated
by the configuration (which {\tt normaliz} is unable to do).  The
computations done with {\tt normaliz} are cached to improve speed.

The input is a point configuration in the usual format and the output is "true" or "false".

\subsection{Graver}
\label{cats_graver}
The program {\tt cats\_graver} can compute the graver basis of a toric
ideal. The input should be a matrix (see \ref{input}).  The program
simply calls 4ti2's {\tt graver} command. The reason for using the
CaTS version of this program is that the input and output are written
in formats that CaTS supports. The Graver basis is part of the input 
to the computation of monomial $A$-graded ideals.

\subsection{BinomialList2Degree}
\label{cats_binomiallist2degree}
The program {\tt cats\_binomiallist2degree} computes the maximal
degree of any binomial in a list of binomials. The program is usually
used in combination with {\tt cats\_graver}.

\subsection{MonomialIdeal2StandardPairs}
\label{cats_monomialideal2standardpairs}
The program {\tt cats\_monomialideal2standardpairs} computes the
standard pairs of a monomial ideal.  Example:
\begin{verbatim}
./cats_monomialideal2standardpairs
\end{verbatim}
followed by
\begin{verbatim}
{a^2*b,b*b}
\end{verbatim}
produces the list
\begin{verbatim}
{( 1 , (a) ),
( a*b , () ),
( b , () )}
\end{verbatim}

%\subsection{HighDegreeFibers}
%This program computes the fibers of any Graver multi-degree. By default it will 
%\subsection{FirlaZiegler}
%Internal use
%\subsection{Conjecture}
%Internal use

\subsection{MaximalMinors}
\label{cats_maximalminors}
The program {\tt cats\_maximalminors} computes the set of determinants
of all maximal minors of the input matrix. The input matrix must have
more columns than rows.
Example:
\begin{verbatim}
./cats_maximalminors
\end{verbatim}.
followed by
\begin{verbatim}
{(0,1)(2,3)(4,5)}
\end{verbatim}
produces the list
\begin{verbatim}
{-4, -2}
\end{verbatim}
So the given point configuration is not unimodular.
