\newpage
\section{Installation}
\label{sec:installation}
If you are using MacOS the easiest way to install Gfan is to use precompiled executables: go to the Gfan webpage, go to the binaries.html subpage, and follow the instructions there.

If you are using Linux the following might work
\begin{verbatim}
sudo apt-get install gfan
\end{verbatim}
or
\begin{verbatim}
sudo emerge gfan
\end{verbatim}
depending on your distribution and package manager. If you succeed, it is good to know which version was installed. Run
\begin{verbatim}
gfan _version
\end{verbatim} 
Should this command fail, then you are using an old version of gfan.


The rest of this section eplains how to install Gfan by compiling it from source on a Linux/Unix-like
system with a modern version of gcc. \name has been compiled
successfully with gcc version~4.1.1. {\bf Two libraries are needed in order
to compile \name: {cddlib} and { gmp}.} Users of Microsoft
Windows may be able to use these installation instructions if they
first install Cygwin. A new feature in \name version 0.4 is the
possibility to link to the SoPlex \cite{wunderling} library. This does
not add to the functionality of \name but improves speed of the
polyhedral computations. In an attempt to keep the installation instructions simple, instructions for how to use SoPlex are given in a separate section, Subsection \ref{subsec:soplex}. {\color{red} If you are a lucky Linux user it will suffice to follow the red part of these instructions.}

\subsection{Installation of the gmp library}
GMP stands for GNU Multi Precision arithmetic library. This library
must be installed on your system before you can install cddlib and
gfan.
%With the
%current {\tt Makefile} of \name gmp must have been installed by the
%root superuser.
{\color{red}
{\bf On \sout{most} some GNU/Linux systems the library is already installed.}}
%\name can be compiled with gmp version 4.1.2.
If your system does not already
have gmp installed (which is the case if you have a usual Mac OS X
installation) follow the directions in this section.

\vspace{0.3cm}
IF YOU ARE USING Mac OS X AND YOU ARE NOT AN EXPERT FOLLOW THE INSTRUCTIONS IN SECTION~\ref{sec:fink} INSTEAD.

\vspace{0.3cm}
\noindent
Make a new directory and download {\tt gmp-4.2.2.tar.gz} from

\centerline{\tt http://gmplib.org/}

\noindent
for example by typing
\begin{verbatim}
cd ~
mkdir tempdir
cd tempdir
wget http://ftp.sunet.se/pub/gnu/gmp/gmp-4.2.2.tar.gz
\end{verbatim}
Extract the file and go to the thereby created directory:
\begin{verbatim}
tar -xzvf gmp-4.2.2.tar.gz
cd gmp-4.2.2
\end{verbatim}
Run the configure script and specify the installation directory:
\begin{verbatim}
./configure --prefix=$HOME/gfan/gmp
\end{verbatim}
%$
The above line specifies the installation directory which in this case
will be the folder {\tt gfan/gmp} in your home directory. If you already have
a directory by that name its content may be destroyed by the subsequent commands.

Compile the gmp library and install it:
\begin{verbatim}
make
make install
\end{verbatim}
Finally, a very important step when working with gmp: Let the program
perform a self-test:
\begin{verbatim}
make check
\end{verbatim}
The gmp installation is now complete. The gmp files can be found in your
home directory under {\tt gfan/gmp}.

\subsubsection{Installing the gmp library on Mac OS X using fink}
\label{sec:fink}
Current versions of Mac OS X and the gmp library have a compatibility
problem causing gmp to be compiled with errors if compiled without
modifications. There exist packages of gmp for Mac OS X on the internet
which have been compiled incorrectly. We recommend that Mac OS users
use the packages provided by fink.

\vspace{0.3cm}

\noindent
Install fink by following the instructions given on the page

\centerline{\tt http://www.finkproject.org/download/index.php?phpLang=en}

\noindent
Having installed fink now simply type
\begin{verbatim}
fink install gmp
\end{verbatim}
The gmp library is now installed in the directory {\tt /sw}.

\subsection{Installation of the cddlib library}
Cddlib \cite{cdd} is a library for doing exact polyhedral
computations, including solving linear programming problems. \name can
be compiled with cddlib version 094. Older versions of cddlib will not
work with \name version 0.2 or later.
% The library can be installed
%anywhere in the file system, so superuser access is not needed.
Notice that cddlib itself needs gmp to compile. We give
instructions on how to install cddlib.
% If this does not work have a look in the cddlib manual.

\vspace{0.3cm}

\noindent
\color{red}
Make a directory for the compilation process if you did not do that already:
\begin{verbatim}
cd ~
mkdir tempdir
cd tempdir
\end{verbatim}

\noindent
Download the file {\tt cddlib-094f.tar.gz} from
\begin{verbatim}
http://www.ifor.math.ethz.ch/~fukuda/cdd_home/cdd.html
\end{verbatim}
% http://www.cs.mcgill.ca/~fukuda/soft/cdd_home/cdd.html
into that directory.
Decompress the file and change directory to the directory being created:
\begin{verbatim}
tar -xzvf cddlib-094f.tar.gz
cd cddlib-094f
\end{verbatim}
Run the configure script. \color{black} Here you have the chance of telling cddlib where to find gmp and where to install itself.
\begin{verbatim}
./configure --prefix="$HOME/gfan/cddlib"
      CFLAGS="-I$HOME/gfan/gmp/include -L$HOME/gfan/gmp/lib"
\end{verbatim}
%$
(On a single line).
The above options say that cddlib should be installed in your home directory under {\tt gfan/cddlib} and where to look for gmp. If gmp was installed by fink (see Section~\ref{sec:fink}) you should run
\begin{verbatim}
./configure --prefix="$HOME/gfan/cddlib"
      CFLAGS="-I/sw/include -L/sw/lib"
\end{verbatim}
instead. {\bf If gmp was already installed on your system in its default location run}
\color{red}
\begin{verbatim}
./configure --prefix="$HOME/gfan/cddlib"
\end{verbatim}
\color{black}
%$
The content of {\tt gfan/cddlib} might be destroyed by the subsequent commands.
Compile and install cddlib:
\color{red}
\begin{verbatim}
make
make install
\end{verbatim}
\color{black}
You can now find the installed cddlib library files in your home directory under {\tt gfan/cddlib}.

\noindent
If you had super user access you could also just have run
\begin{verbatim}
./configure
\end{verbatim}
when you configured cddlib. This would cause cddlib to be installed in its default place.

\subsection{\name installation}
\label{subsec:installation}
\color{red}
Download the file {\tt \nameversion .tar.gz} from the \name
homepage located at:
% \centerline{\tt http://www.soopadoopa.dk/anders/gfan/gfan.html.}
% \centerline{\tt http://home.imf.au.dk/ajensen/software/gfan/gfan.html}
\begin{verbatim}
http://www.math.tu-berlin.de/~jensen/software/gfan/gfan.html
\end{verbatim}
to your folder {\texttt tempdir}.
\noindent
Extract the file and enter the new directory by typing
\begin{alltt}
cd ~
cd tempdir
tar -xzvf \nameversion.tar.gz
cd \nameversion
\end{alltt}
\color{black}
Gfan does not have a configure script, so you tell Gfan where to find gmp and cdd when you compile the program. For example you should type
\begin{verbatim}
make
\end{verbatim}
or

\color{red}
\begin{verbatim}
make cddpath=$HOME/gfan/cddlib
\end{verbatim}
\color{black}
%$
or
\begin{verbatim}
make cddpath=$HOME/gfan/cddlib gmppath=$HOME/gfan/gmp
\end{verbatim}
%$
or
\begin{verbatim}
make cddpath=$HOME/gfan/cddlib gmppath=/sw
\end{verbatim}
%$
depending on where you installed the libraries to compile the program.

The final step is to install the compiled program. Type
\color{red}
\begin{verbatim}
make PREFIX=$HOME/gfan install
\end{verbatim}
%$
\color{black}or
\begin{verbatim}
make install
\end{verbatim}
depending on where you want Gfan installed. (The second line attempts to install it in {\tt /usr/local} by default). If you chose to install in
the directory {\tt gfan} in your home folder \color{red}you will now find the
file {\tt gfan} in the subdirectory {\tt gfan/bin} of your home folder together with a set of symbolic links\color{black},
for example {\tt gfan\_buchberger}.
You can go to the subdirectory and type {\tt ./\exename{} --help} and {\tt ./\exename{}\_buchberger --help} in
the shell to test them. Or you can ask Gfan to compute the reduced Gr\"obner bases of an ideal by typing
\begin{alltt}
./\exename{}\_bases
\end{alltt}
followed by, for example,
\begin{verbatim}
Q[a,b,c]
{a^3+b^2c-a,c^2-2/3b}
\end{verbatim}
\begin{remark}
If for some reason you did get {\tt gfan} compiled but did not get the symbolic links made like {\tt gfan\_buchberger} you can still run that program by typing {\tt gfan \_buchberger} instead of {\tt gfan\_buchberger}.
\end{remark}


\subsection{SoPlex (for the advanced user only)}
\label{subsec:soplex}
Linking Gfan to SoPlex can lead to huge performance
improvements. Notice however, that the strict license of SoPlex
propagates through the software to your paper, requiring that you cite
SoPlex appropriately if you choose to publish results based on SoPlex.
Furthermore, with the standard SoPlex license you are only allowed to
use SoPlex for non-commercial, academic work.

Download SoPlex here (version 1.3.2 has been used successfully):
\begin{verbatim}
http://soplex.zib.de/download.shtml
\end{verbatim}
After download, follow the installation instructions
\begin{verbatim}
http://www.zib.de/Optimization/Software/Soplex/html/INST.html
\end{verbatim}
After having installed SoPlex, you must tell \name where SoPlex is located. Do this by editing the lines
\begin{footnotesize}
\begin{verbatim}
SOPLEX_PATH = $(HOME)/math/software/soplex-1.3.2
SOPLEX_LINKOPTIONS = -lz $(SOPLEX_PATH)/lib/libsoplex.darwin.x86.gnu.opt.a
\end{verbatim}
\end{footnotesize}
of the file \texttt{Makefile} in your \name directory.  Most likely
you need to change \texttt{darwin} to \texttt{linux} in the last line.
Finally you need to recompile \name. First run \texttt{make clean} and
then \texttt{make} with the options from
Subsection~\ref{subsec:installation} together with the
option \texttt{soplex=true}. Then do a \texttt{make install} as
described in Subsection~\ref{subsec:installation}.

%Please keep on reading --- in the next section we will see how to install all the additional \name programs.

%\subsection{Installation to invoke additional features}
%If you have root access to the system it is recommended that you type
%\begin{verbatim}
%make install
%\end{verbatim}
%in your shell after having compiled \name. This will copy the file {\tt \exename} to the directory {\tt /usr/local/bin} and \name will now be accessible from any directory and by any user by typing
%\begin{alltt}
%\exename
%\end{alltt} 
%The {\tt make install} step above also creates symbolic links for the additional programs included in \name (like {\tt \exename \_buchberger} or {\tt \exename \_render}). If you chose not to run {\tt make install} you can install the additional programs in \name by typing
%\begin{alltt}
%./\exename installlinks
%\end{alltt}
%This will create a set of symbolic links to {\tt \exename} in the current directory. Invoking {\tt \exename} with one of these new names will have a different meaning. For example you may use {\tt \exename \_buchberger} to compute a single Gr\"obner basis. These programs have a help file which can be displayed by invoking the programs with the option {\tt --help}. The contents of the help files are also listed in Section \ref{sec:applist}.



%\subsection{MacOS}

%\subsubsection{Installing gmp from source}


%\subsubsection{Installing with fink}
%\begin{description}
%\item{gmp} dgdgdfdg

%\item{cddlib}
%Download the file {\tt cddlib-094b.tar.gz} from
%\begin{verbatim}
%http://www.cs.mcgill.ca/~fukuda/soft/cdd_home/cdd.html
%\end{verbatim}
%into a directory. Decompress the file using {\tt gzip -d
%  cddlib-094b.tar.gz} and extract the tar archive using {\tt tar -xvf
%  cddlib-094b.tar}. Change directory to the newly created directory
%{\tt cddlib-094b} and run
%\begin{verbatim}
%./configure --prefix="$HOME/cddlib" CFLAGS="-I/sw/include -L/sw/lib"
%\end{verbatim}
%When the following command is run the directory ``cddlib'' is created in your home directory. If it already exists the directory is destroyed.
%\begin{verbatim}
%make install
%\end{verbatim}
%\end{description}



%\begin{center} 
%  {\bf The complete list of functionalities available in CaTS and the
%    programs they need are listed at the end of the manual.}
%\end{center}







